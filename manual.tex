\documentclass[a4paper, 11pt]{article}
\usepackage[czech]{babel}
\usepackage{times}
\usepackage{geometry}
\usepackage[utf8]{inputenc}
\usepackage[T1]{fontenc}
\usepackage{graphicx}
\usepackage{hyperref}

\hypersetup{colorlinks = true, hypertexnames = false}

\geometry{
	a4paper,
	total={170mm, 240mm,},
	left = 20mm,
	top = 30mm,
}

\begin{document}
\begin{titlepage}
	\begin{center}
		\Huge
		\textsc{Vysoké učení technické v~Brně} \\
		\huge
		\textsc{Fakulta informačních technologií} \\
		\vspace{\stretch{0.382}}
		\LARGE
		Síťové aplikace a~správa sítí\\
		\Huge
		Přenos souboru skrz skrztý kanál
		\vspace{\stretch{0.618}}
	\end{center}
	{\Large
		\today
		\hfill
		Marek Hlávka (xhlavk09)
	}
\end{titlepage}
\tableofcontents
\newpage

\section{Zadání}
Zadáním projektu bylo vytvořit klient/server aplikaci, pomocí které bude možno přenést soubor skrz skrytý kanál, kde jsou data přenášena pomocí ICMP protokolu, resp. uvnitř ICMP Echo-Request/Response zpráv. Kvůli zabezpečení musí být soubor zašifrován pomocí šifry AES. Serverová strana aplikace pak bude ukládat soubor do složky, ve které je spuštěn.

\section{Spuštění}

\centerline{\large./secret -s <ip|hostname> -r <file> [-l]}
\begin{itemize}
\item -s <ip|hostname> : adresa nebo hostname serveru na který bude odeslaný soubor
\item -r <file> : specifikace souboru pro přenos
\item -l : program spuštěný s tímto parametrem se chová jako server, který přijímá soubory posílané od klientů
\end{itemize}
Příklad souboru s uloženými servry pro filtraci:\\
\begin{figure}[h]
%%%\includegraphics[scale=1.1]{}
\end{figure}

\newpage
\section{Návrh}
Jako jazyk k implemntaci byl zvolen jazyk C. Celý program je rozdělen do dvou větších celků: 1. Strana klienta a 2. Strana serveru

\section{Argunemty}
V této části je kontrolováno spuštění programu, resp. kontrola argumentů při spuštění. K tomuto účelu je využita funkce \textit{getopt()} z knihovny \textit{getopt.h}. Jsou postupně zkontrolovány parametry, které jsou blíže popsané v sekci \textit{Spuštění}. Pokud nejsou některé požadované parametry uvedeny, program vypíše nápovědu a úspěšně se ukončí. V případě zadání parametru \textit{-l} nesjou vyžadovány další argumenty, vzhledem k tomu že server nepotřebuje cílovou ip adresu ani specifikovat souor pro přenos.
\section{Klient}

\subsection{Spuštení zachytávání paketů}
Pro zachytávání paketů ja využita knihovna \textit{libpcap.h}, která umožnujě poslouchat pakety na příslušném rozhraní a s příslušným filtrem. Díky tomu můžeme zpracovávat pakety, které jsou jen v UDP komunikaci a na portu buď zadaném uživatelem nebo defaultně na portu 53. Protože v zadáná nebylo upřesněno na jakém rozhraní má program poslouchat pakety, je předpoklad že je program má očekávat na kterémkoliv rozhraní. Z tohoto důvodu si program nejdříve zjistí všechny rozhraní zařízení na kterém je spuštěn. Poté si pro daný počet rozhraní vytvoří separátní vlákna, pro každé rozhraní jedno. Pro zajištění správného přístupu k některým datům jsou vytvořeny semafory, které zajistí, že data nejsou změneny zatímco s nimi pracuje jiné vlákno. Následně je pomocí funkce \textit{pcap\_open\_live()} otevřena relace na daném rozhraní, poté je na něj aplikován filtr jako poslední je zavolána funkce \textit{pcap\_loop()} pro cyklické poslouchání paketů a jejich následné zpracování. Po skončení relace jsou uvolněné zdroje, teré byly dříve rezervované programem. Samotný program je poté ukončen sticknutím zkratky \textit{Ctrl + C}.

\subsection{Zpracování paketů}
Zpracování paketů probíhá ve funkci \textit{got\_packet()}, která je v souboru \textit{rec\_pack.c}. Celý postup spočívá v tom zpracovávat postupně hlavičky jednotlivých protokolů až se program dostane k samotnému dns paketu a jeho data následně zpracuje.

\section{Server}

\subsubsection{Rozklad paketů}
 Jako první je hlavička IPv4 resp. IPv6. Pomocí funkce \textit{ether\_packet()} je určeno jestli je právě zpracovaný paket IPv4 nebo IPv6. Nálsedně je pomocí funkcí \textit{ip\_handle()} a \textit{ipv6\_handle()} uložena IP hlavička a její velikost stejně jao hlavička UDP. následně je pakte bez těchto hlaviček předán funkci \textit{dns\_handle()}, kde je už zpracovávána samotný dns paket. Do jednotlivých proměnných jsou uolžené hodnoty z dns hlavičky jako např. ID, RCODE nebo počet odpovědí, které jsou v paketu obsaženy. Následně je dotaz obsažený v paketu otestován jestli má sptrávnou třídu a typ a v neposldní ředě je zkontrolována adresa obsažena v dotazu, jestli je obsažena v souboru dodaném argumentem programu. Pokud je adresa v tomto souboru je nutné odelat tazateli zpět dns paket s vhodným RCODEm pomocí funkce\textit{bad\_response()}, jinak je paket přeposlaný dál pomocí funkce \textit{pkt\_forward()}.
 
\subsubsection{Filtrované pakety}
Na filtrované pakety, resp. pakety jejichž hostaame je obsaženo v souboru \textit{hostname\_file}, jsou tvořeny odpovědi ve funkci \textit{bad\_response}. Nejdřive je nastavena dns hlavička odesílaného paketu:\\
\begin{figure}[h]
%%%\includegraphics[scale=1.1]{}
\end{figure}\\
Následně je otevřený socket pomocí knihovny \textit{sys/socket.h} a ja nasteven podle toho jestli program vrací IPv4 paket nebo IPv6. Poté je jednoduše pomocí funkce \textit{sendto()} odeslaná zpráva obsahující dns paket s potřebnými informacemi zpět tazateli.

\subsubsection{Ostatní pakety}
Ostatní pakety jsou přeposílány neznámému serveru na adrese získananým argumentem \textit{-s server}. Pakety jsou přeposílány v nezměněné podobě s původní hlavičkou ze získaného paketu. Stejně jako u filtrovaných paketů je nastevan socket na následně pomocí funkcí \textit{sendto} a \textit{recvfrom} je tento paket odeslaný a je na něj zísána odpověď, která je poté přeposlána původnímu tazateli. Před samotným odesláním odpovědi je socket přenastaven podle toho jestli vracíme IPv4 nebo IPv6 odověď.
\end{document}